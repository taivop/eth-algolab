\documentclass[a4paper,titlepage]{article}

\usepackage{geometry}
\geometry{a4paper, left=10mm, right=10mm, top=10mm, bottom=15mm}

\usepackage[colorlinks=false]{hyperref}
\usepackage[utf8]{inputenc}
\usepackage{float}
\usepackage{pdfpages}
\usepackage{imakeidx}
%\usepackage{pdflscape}
\usepackage{graphicx}

\usepackage{listings}
\lstset{
	language=C++,
	basicstyle=\scriptsize\ttfamily,
    keywordstyle=\color{blue}\ttfamily,
    stringstyle=\color{red}\ttfamily,
    commentstyle=\color{red}\ttfamily,
    morecomment=[l][\color{red}]{\#},
    tabsize=2,
    frame=single,
    inputencoding=utf8,
    extendedchars=true,
    showstringspaces=false,
    breaklines=true
}

\lstset{literate=
  {á}{{\'a}}1 {é}{{\'e}}1 {í}{{\'i}}1 {ó}{{\'o}}1 {ú}{{\'u}}1
  {Á}{{\'A}}1 {É}{{\'E}}1 {Í}{{\'I}}1 {Ó}{{\'O}}1 {Ú}{{\'U}}1
  {à}{{\`a}}1 {è}{{\`e}}1 {ì}{{\`i}}1 {ò}{{\`o}}1 {ù}{{\`u}}1
  {À}{{\`A}}1 {È}{{\'E}}1 {Ì}{{\`I}}1 {Ò}{{\`O}}1 {Ù}{{\`U}}1
  {ä}{{\"a}}1 {ë}{{\"e}}1 {ï}{{\"i}}1 {ö}{{\"o}}1 {ü}{{\"u}}1
  {Ä}{{\"A}}1 {Ë}{{\"E}}1 {Ï}{{\"I}}1 {Ö}{{\"O}}1 {Ü}{{\"U}}1
  {â}{{\^a}}1 {ê}{{\^e}}1 {î}{{\^i}}1 {ô}{{\^o}}1 {û}{{\^u}}1
  {Â}{{\^A}}1 {Ê}{{\^E}}1 {Î}{{\^I}}1 {Ô}{{\^O}}1 {Û}{{\^U}}1
  {œ}{{\oe}}1 {Œ}{{\OE}}1 {æ}{{\ae}}1 {Æ}{{\AE}}1 {ß}{{\ss}}1
  {ç}{{\c c}}1 {Ç}{{\c C}}1 {ø}{{\o}}1 {å}{{\r a}}1 {Å}{{\r A}}1
  {€}{{\EUR}}1 {£}{{\pounds}}1
}

% remove the paragraph indentation
\setlength{\parindent}{0in}

\begin{document}

\section{Most important stuff}

Estonian keymap: \verb+setxkbmap ee+

printf/stdio:
\begin{lstlisting}
std::ios_base::sync_with_stdio(false);
\end{lstlisting}


\subsection{Tricks}
\begin{itemize}
  \item Graphs
    \begin{itemize}
    \item make 2 nodes for every node in original graph (Odd routes)
    \item transform graph into another graph with more/less nodes (Odd routes, Carsharing)
    \item flows: add offset to negative weights so we can use min cost max flow
    \item flows: compress unneeded nodes (long paths) into single nodes (Carsharing)
    \item show that graph is a special case, e.g. bipartite
    \item maximum matching on a tree == vertex-disjoint edges on a tree
      \begin{itemize}
      \item greedily repeat: take an edge which contains a leaf and remove its endpoints 
      \end{itemize}
    \item create a bipartite graph if possible -- much more feasible to do matching and vertex cover on it!
    \item if there is a restriction that each node can be traversed only once, then create in-node and out-node for each node, and do maximum matching on this BIPARTITE graph (Consecutive constructions)
    \end{itemize}
  \item Geometry / graphs on plane
    \begin{itemize} 
    \item good chance is that I can use Delaunay to build quickly if I need proximity structure only
    \item use only predicates if at all possible -- orientation tests can do surprisingly many things
    \end{itemize}
  \item Other
    \begin{itemize}
    \item for interval scheduling, greedy with earliest finish time works!
    \item think whether a brute-force solution would be good enough -- in Stamps it was
    \item chessboard tasks -- use chessboard coloring
    \item find a property of the task that allows me to apply some simpler/better running time algorithm
    \item if I can find nothing else, start from modelling brute force solution and modifying that
    \item if I have many queries, precompute/remember stuff so each query takes short time
    \item note the numerical bounds on inputs
      \begin{itemize}
      \item can I brute force?
      \item can I hold $input$ and $input^2$ in an int/long/double?
      \end{itemize}
    \item if maximising some cost function, think about LP/QP
    \end{itemize}
\end{itemize}

\subsection{Debugging with Eclipse}
For CGAL, need to have \verb+create_cgal_cmake_script+ done first.
\begin{itemize}
\item Start Eclipse (Win-key -- type 'Eclipse'); not the C/C++ debug one
\item Open .cpp file
\item Start debug: Run--Debug OR F11 OR bug-picture button
\item Debugging:
\begin{itemize}
  \item Step into (F5): start stepping through the method call on the current line.
  \item Step over (F6): consider the current method as one step and proceed to next line.
  \item Step return (F7): run the entire method until the end and return.
\end{itemize}
\item Adding input file:
\begin{itemize}
  \item Run [or Debug] Configurations -- Common -- Standard Input and Output
  \item Workspace/File System depending on where file is
\end{itemize}
%\item 
\end{itemize}

\newpage\subsection{Debugging with gdb}
compile with -g flag:
\begin{verbatim}
g++ -g first.cpp -o first.o
gdb first.o
\end{verbatim}
\begin{itemize}
\item b main - Puts a breakpoint at the beginning of the program
\item b - Puts a breakpoint at the current line
\item b N - Puts a breakpoint at line N
\item b +N - Puts a breakpoint N lines down from the current line
\item b fn - Puts a breakpoint at the beginning of function "fn"
\item d N - Deletes breakpoint number N
\item info break - list breakpoints
\item r - Runs the program until a breakpoint or error
\item c - Continues running the program until the next breakpoint or error
\item f - Runs until the current function is finished
\item s - Runs the next line of the program
\item s N - Runs the next N lines of the program
\item n - Like s, but it does not step into functions
\item u N - Runs until you get N lines in front of the current line
\item p var - Prints the current value of the variable "var"
\item bt - Prints a stack trace
\item u - Goes up a level in the stack
\item d - Goes down a level in the stack
\item q - Quits gdb
\end{itemize}


\newpage\section{Templates \& library-specific}
\subsection{ACM}
\lstinputlisting{../template/first.cpp}
\subsection{BGL}
Getting vertex descriptor from vertex ID in BGL: \verb+vertex(vertex_id, G)+
\lstinputlisting{../template-bgl/first.cpp}
\subsection{CGAL}
\lstinputlisting{../template-cgal/first.cpp}
Compiling:
\lstinputlisting{../template-cgal/build.sh}

\newpage\section{Standard stuff}
\subsection{Passing variables by reference}
\lstinputlisting{../chunks/std/pass_by_reference.cpp}
\subsection{Priority queue}
\lstinputlisting{../chunks/std/priority_queue.cpp}
\subsection{Custom structs (and queues)}
\lstinputlisting{../chunks/std/custom_struct.cpp}


\newpage\section{Chunks}
\subsection*{CGAL elementary}
Predicates
\lstinputlisting{../chunks/cgal/elementary/predicates.cpp}
Constructions
\lstinputlisting{../chunks/cgal/elementary/constructions.cpp}
\subsection*{CGAL advanced}
Delauney-Voronoi
\lstinputlisting{../chunks/cgal/advanced/delauney-voronoi.cpp}
\subsection*{BGL elementary}
Matching
\lstinputlisting{../chunks/bgl/elementary/matching.cpp}
Minimum spanning tree (MST) and strongly connected components (SCC)
\lstinputlisting{../chunks/bgl/elementary/mst_scc.cpp}
Shortest paths
\lstinputlisting{../chunks/bgl/elementary/shortest_paths.cpp}
\subsection*{BGL advanced (flows)}
Simple flows
\lstinputlisting{../chunks/bgl/flows/simple_flows.cpp}
Bipartite matching
\lstinputlisting{../chunks/bgl/flows/bipartite_matching.cpp}
Minimum cut
\lstinputlisting{../chunks/bgl/flows/min_cut.cpp}
Min cost max flow
\lstinputlisting{../chunks/bgl/flows/min_cost_max_flow.cpp}

\newpage\section{My solutions}

\subsection*{1-Build the sum}
\lstinputlisting{../1-buildthesum/first.cpp}
\subsection*{1-Dominoes}
\lstinputlisting{../1-dominoes/first.cpp}
\subsection*{1-Shelves}
\lstinputlisting{../1-shelves/fourth.cpp}
\subsection*{2-Race tracks}
\lstinputlisting{../2-racetracks/first.cpp}
\subsection*{3-Antenna}
\lstinputlisting{../3-antenna/first.cpp}
\subsection*{3-Hit}
\lstinputlisting{../3-hit/first.cpp}
\subsection*{4-Important bridges}
\lstinputlisting{../4-importantbridges/second.cpp}
\subsection*{6-Coin tossing}
\lstinputlisting{../6-cointossing/first.cpp}
\subsection*{6-Shopping trip}
\lstinputlisting{../6-shoppingtrip/first.cpp}
\subsection*{7-Diet}
\lstinputlisting{../7-diet/first.cpp}
\subsection*{7-Portfolios}
\lstinputlisting{../7-portfolios/first.cpp}
\subsection*{8-Bistro}
\lstinputlisting{../8-bistro/first.cpp}
\subsection*{8-Graypes}
\lstinputlisting{../8-graypes/second.cpp}

%\subsection*{}
%\lstinputlisting{..//first.cpp}




\end{document}



